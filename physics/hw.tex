\documentclass[12pt,a4paper]{report}
% \usepackage[utf8]{inputenc}
% \usepackage[vietnam]{babel}
\usepackage[utf8]{vietnam}
\usepackage{amsmath}
\usepackage{amsfonts}
\usepackage{amssymb}
\usepackage{graphicx}
\usepackage{color}
\usepackage{framed}
\usepackage[left=2cm,right=2cm,top=2cm,bottom=2cm]{geometry}
\usepackage{tikz,tkz-tab}


\title{\framebox {
        \textcolor{TEcolor}{
            \Huge {    AP/College Physics 1    }
        }
    }    }
    
\author{\Large @arch-techs}
\date{2021}

\definecolor{TEcolor}{RGB}{0, 50, 50}

\usepackage{fancyhdr}

\pagestyle{fancy}
\fancyhf{}
\lhead{\includegraphics[scale=0.2]{TE1}
\textcolor{TEcolor}{
\fontfamily{cmss}\selectfont
@arch-techs}
}
\rhead{\textcolor{TEcolor} {
	\fontfamily{cmss}\selectfont AP/College Physics 1
}}
\rfoot{
\fontfamily{cmss}\selectfont \textcolor{TEcolor}{
Page \thepage}}


\begin{document}
{\fontfamily{cmss}\selectfont

\begin{center}
    \framebox{
      \Large \textcolor{TEcolor}{BTVN: Chương 5: Nhiệt Động Lực Học Chất Khí}
}
\newline

      \textcolor{TEcolor}{Họ và tên: Nguyễn Đình Hoàng - 20021361}
\end{center}

Bài 3: Thể tích của một khối lượng nhất định khí lý tưởng khi được nung nóng
$\varDelta T$ = 1 K ở áp suất không đổi tăng thêm $\alpha $ = $\dfrac{1}{335}$ phần của thể tích ban đầu.
Nhiệt độ khí ban đầu là bao nhiêu?
\begin{center}
    Bài làm: \\
    Do áp suất không đổi, nên ta có phươn trình: \\
    $\dfrac{V_{1}}{T_{1}} = \dfrac{V_{2}}{T_{2}}$
    $\Leftrightarrow  \dfrac{V_{1}}{T_{1}} = \dfrac{V_{1} + \dfrac{1}{335}V_{1}}{T_{1} + 1}$
    $\Leftrightarrow  \dfrac{1}{T_{1}} = \dfrac{1 + \dfrac{1}{335}}{T_{1} + 1}$
    \[\Rightarrow T_{1} = 335K\] 
    Vậy nhiệt độ khí ban đầu là: $335K$.

\end{center}

Bài 4: Chất khí ở $t_{1} = 107^o C $ có áp suất $p_{1} = 10^6 Pa.$ Áp suất khí sẽ giảm đi bao
nhiêu khi nó được làm lạnh đẳng tích đến $t_{2} = -13^o C$?
\begin{center}
    Bài làm: \\
    Do là quá trình đẳng tích nên ta có phươn trình: \\
    $\dfrac{P_{1}}{T_{1}} = \dfrac{P_{2}}{T_{2}}$
    $\Leftrightarrow  \dfrac{10^6}{107 + 273} = \dfrac{P_{2}}{-13 + 273}$
    \[\Rightarrow P_{2} \thickapprox 0,684.10^6 Pa\] 
    Vậy áp suất khí sẽ giảm đi: $10^6 - 0,684.10^6 = 3,16.10^5 (Pa)$.

\end{center}

Bài 5: Một hình trụ chứa thể tích $V_{1} = 0,035 m^3$ không khí dưới áp suất $p_{1} = 100kPa$, nối với hình trụ có thể tích $V_{2} = 0,015 m^3$
từ đó không khí được bơm ra hoàn toàn (bình chân không). Tìm áp suất $p_{2}$ được thiết lập trong các bình sau
khi chúng được nối với nhau biết nhiệt độ không đổi.

\begin{center}
    Ta có: $V_{1} = 0,035 m^3 = 35l$, $V_{2} = 0,015 m^3 = 15l$. \\
    Do quá trình đẳng nhiệt nên ta có phương trình: \\
    $P_{1}V_{1} = P_{2}V_{2}$
    $\Leftrightarrow 100(kPa).35 = P_{2}.(15 + 35)$ \\
    $\Rightarrow P_{2} = 70 (kPa)$
\end{center}

Bài 6: Cho giản đồ $V(T)$ của một khí lý tưởng như hình:


\begin{center}
    \begin{tikzpicture}[scale=1, font=\footnotesize, line join=round, line cap=round, >=stealth]
    \draw (0,0) node [below left] {$O$};
    \draw[->] (0,0)--(5,0) node[below] {$T$};
    \draw[->] (0,0)--(0,5) node[below left] {$V$};
    {\color{black}}{
    \draw[samples=200,domain=1:3,smooth,variable=\x] plot (\x,{(\x)}) node[right]{};
    }
    \draw[samples=200,domain=0:1,smooth,variable=\x, dashed] plot (\x,{(\x)}) node[right]{};
    \draw[-](3,3)|-(3,2) (3,2) |- (4,2);
    \path
    (1,1)node[below]{$1$}
    (3,3)node[above]{$2$}
    (3,2)node[below]{$3$}
    (4,2)node[below]{$4$};
    \end{tikzpicture}
\end{center}
 Hãy vẽ giản đồ p-V và p-T cho các quá trình trên.
\newpage

\begin{center}
    \begin{tikzpicture}[scale=1, font=\footnotesize, line join=round, line cap=round, >=stealth]
    \draw (0,0) node [below left] {$O$};
    \draw[->] (0,0)--(0,5) node[below left] {$P$};
    \draw[->] (0,0)--(5,0) node[below] {$V$};
    \draw[samples=200,domain=4:2.5,smooth,variable=\x] plot (\x,{-1*(\x)+ 5}) node[right]{};
    \draw[-](1,1)|-(4,1) (2.5,2.5) |- (2.5,4);
    \path
    (1,1)node[below]{$1$}
    (4,1)node[below]{$2$}
    (2.5,2.5)node[below]{$3$}
    (2.5,4)node[above]{$4$};
    \end{tikzpicture}
    \hspace{3cm}
    \begin{tikzpicture}[scale=1, font=\footnotesize, line join=round, line cap=round, >=stealth]
        \draw (0,0) node [below left] {$O$};
        \draw[->] (0,0)--(0,5) node[below left] {$P$};
        \draw[->] (0,0)--(5,0) node[below] {$T$};
        \draw[samples=200,domain=3:4.5,smooth,variable=\x] plot (\x,{(\x)-1/2}) node[right]{};
        \draw[-](1,1)|-(3,1) (3,1) |- (3,2.5);
        \path
        (1,1)node[below]{$1$}
        (3,1)node[below]{$2$}
        (3,2.5)node[left]{$3$}
        (4.5,4)node[above]{$4$};
        \end{tikzpicture}
        \\
        \vspace{5cm}
        THE END
\end{center}
}
\end{document}